\documentclass{article}
\usepackage{minted}


\author{Manuel Flückiger 22-112-502}
\title{Rechnerarchitektur Serie 2}
\date{\today}

\begin{document}

\newcommand{\mycomment}[1]{}

\maketitle

\section*{Theorie Aufgaben}
\section{Integrated Circuit}
\begin{itemize}
    \item Increased processing power. \newline
    Miniaturization enables more components to be integrated into a 
    single wafer, which increases the number of connections between 
    the integrated circuit and printed circuit board. This increases 
    the ability to route signals and therefore increases the processing 
    power of the system it is being assembled into

    \item Reduced power consumption. \newline
    Miniaturization reduces the distance between components on a chip, 
    which reduces the amount of power required to move data between them.
\end{itemize}

\section{Influence on the conductivity of semiconductors during production}

The conductivity of a semiconductor can be influenced by doping during production. 
Doping is the process of adding impurities to a pure semiconductor material to change
its electrical properties. Very small amounts of dopants (in the parts-per-million range)
dramatically affect the conductivity of semiconductors. The electrical conductivity
of a material depends on the number of free electrons and holes per unit volume and 
on the rate at which these carriers move under the influence of
an electric field. In an intrinsic semiconductor there exists an equal number of 
free electrons and holes. If the temperature of the semiconductor increases, the 
concentration of charge carriers (electrons and holes) is also increased. Hence, 
the conductivity of a semiconductor is increased accordingly. 

N-doping and P-doping are two types of doping used in semiconductors.
N-type doping is the process of adding impurities such as phosphorus or arsenic
to a pure semiconductor material to increase the number of free electrons in the 
material. P-type doping is the process of adding impurities such as boron or aluminum 
to a pure semiconductor material to increase the number of holes in the material. Doped
semiconductors are electrically neutral. The terms n- and p-type doped do only refer to 
the majority charge carriers. Each positive or negative charge carrier belongs to a fixed
negative or positive charged dopant. N- and p-doped semiconductors behave approximately 
equal in relation to the current flow.

\section{Changing the conductivity of a MOSFET}

The conductivity of an NMOS MOSFET between drain and source changes based on the voltage
applied to the gate terminal. The gate terminal is separated from the channel by a thin
layer of oxide which acts as a dielectric material. When a voltage is applied to the
gate terminal, an electric field is created across the oxide layer which induces a 
channel between the source and drain terminals. The conductivity of this channel depends
on the voltage applied to the gate terminal and can be used for amplifying or switching.


\section{MOSFET characteristics}

When the gate voltage of an NMOS MOSFET is above the threshold voltage, it enters 
into saturation region where the current between drain and source becomes almost 
constant with increasing drain-source voltage.


\section{Performance calculations}
\begin{itemize}
    \item{How much faster/sloqer is a machine that needs 6 clock cycles for the LOAD instructions}\newline
    Assuming all operations are performed around the same amount:
    \newline
    $ LOAD\ with\ 4\ cycles \rightarrow execution\ of\ all\ operations\ takes\ 15\ cycles. $\newline
    $ LOAD\ with\ 6\ cycles \rightarrow execution\ of\ all\ operations\ takes\ 17\ cycles. $\newline
    $ cycles_{old} = \frac{15}{15}=100\%, cycles_{new} = \frac{17}{15}=113.3\%$ \newline
    that means on average you need 13.3\% more time to complete a programme.

    \item{How much faster is a CPU when its STORE works twice as fast?}\newline
    Assuming all operations are performed around the same amount:
    \newline
    $ STORE\ with\ 12\ cycles \rightarrow execution\ of\ all\ operations\ takes\ 15\ cycles. $\newline
    $ STORE\ with\ 6\ cycles \rightarrow execution\ of\ all\ operations\ takes\ 12\ cycles. $\newline
    $ cycles_{old} = \frac{15}{15}=100\%, cycles_{new} = \frac{12}{15}=80\%$ \newline
    that means on average you need 20\% less time to complete a programme.
\end{itemize}

\section{Using a stack in subroutines}
There are several reasons why one needs the stack for assembler subroutines. 
One reason is that it allows for nested subroutines. Another reason is that 
it allows for passing parameters to subroutines. The processor stack is used 
to store the return address of the subroutine.

\section{ALU & SLT}
\begin{itemize}
    \item The slt command in the ALU (Arithmetic Logic Unit) stands for “Set Less Than”. 
    It is a command that compares two values and sets a register to 1 if the first value 
    is less than the second value, and 0 otherwise.
    \item The ALU calculates that using second
    value minus first value and if the result is $< 0$, the msb (most significant bit) will
    be 1 and the first value must be smaller than the second value. This will then be fed back 
    to the ALU and will be the output.
\end{itemize}

\end{document}
